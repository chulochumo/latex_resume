%% start of file `ed-resume.tex'.

%% Copyright 2006-2012 Xavier Danaux (xdanaux@gmail.com).
%
% This work may be distributed and/or modified under the
% conditions of the LaTeX Project Public License version 1.3c,
% available at http://www.latex-project.org/lppl/.


\documentclass[11p,letterpaper,sans]{moderncv}

\moderncvstyle{classic}
\moderncvcolor{green}  
\renewcommand{\familydefault}{\sfdefault}  
\nopagenumbers{}      

% adjust the page margins
\usepackage[scale=0.92]{geometry}
\setlength{\hintscolumnwidth}{2cm}        
\setlength{\maketitlenamewidth}{9cm}      

\usepackage{cite}

\firstname{Eduardo}
\familyname{Gonzalez}
%\title{}  
\address{11725 Raymond C Ewry ln}{Austin, TX 78748}
\mobile{+1~(979)~665~6838}                     
\email{chulochumo@gmail.com}                   
\homepage{chulochumo.github.com}               

% to show numerical labels in the bibliography (default is to show no
% labels); only useful if you make citations in your resume
\makeatletter
\renewcommand*{\bibliographyitemlabel}{\@biblabel{\arabic{enumiv}}}
\makeatother
%----------------------------------------------------------------------------------
%            content
%----------------------------------------------------------------------------------
\begin{document}
\maketitle
\vspace*{-2\baselineskip}
\section{Education}
\cventry{2011 - 2014}{M.S in Software Engineering}{Texas
  State University}{San Marcos, TX}{}{} 
\cventry{2007  - 2011}{B.S in Electrical Engineering}{Texas State Unversity}{San
  Marcos, TX}{}{Specialized in Communication Systems
  and Networks}

\section{Master thesis}
\cvitem{title}{\emph{Reducing SCTP's (Stream Control Transmission Protocol) Time-to-Complete by Manipulation of its Retransmission Time Out Minimum \cite{EdThesis}}} \cvitem{advisors}{Dr. Stan McClellan, Dr. Wuxu
  Peng, Dr. Mina Guirguis} \cvitem{description}{The Stream Control Transmission Protocol (SCTP) is a relatively young transport protocol that was originally designed to transfer SS7 signaling messages over packet switched networks but has since been standardized for general use. This thesis focuses on SCTP's retransmission mechanisms and how they're affected by network conditions, providing insight into optimization opportunities. For this purpose a research platform is presented that enables rapid prototyping of new algorithms and fast turnover of performance data which is then used to verify previous SCTP research. Finally, we present a novel algorithm for dynamically determining the optimum Retransmission Time Out Minimum (RTOmin) value of an SCTP association that significantly improves performance while avoiding spurious retransmissions.}

\section{Computer skills}
\cvitem{basic}{Python, HTML, Java, Javascript, Linux Kernel, XML,
CMake, Autotools, Qt, }
\cvitem{intermediate}{MATLAB/Simulink, C++, Labview,
  MultiSim, TCP/IP, Linux Development, Lab Testing Equipment,
  WireShark, Git}
\cvitem{advanced}{Assembly(Motorola 68k), SCTP, C, Shell}

\bibliographystyle{IEEEbib}
\bibliography{IEEEabrv,abbrevs,EdResume}

\section{Experience}

\cventry{Aug 2013 - Present}
{Software Engineer}
{FlexRadio Systems}
{Austin, TX}
{}
{
Part of an interdisciplinary team writing software for high performance networked radio transceivers and one of several ``full-stack'' software engineers working with u-boot, Embedded Linux, DSP algorithms, network control, and user facing GUI components \cite{DCCPaper}.\\
Main duties include:
\begin{itemize}
\item Maintaining embedded Linux Kernel and modules
\item Maintaining Audio CODEC binaries and modules
\item Embedded software specification, design and implementation 
\item Improving software development tools and workflow
\item Writing and maintaining Windows WPF GUI components (C\#)
\end{itemize}
}

\cventry{Apr - Jul 2013}
{Contract Software Engineer}
{FlexRadio Systems}
{Austin, TX}
{}
{
\begin{itemize}
\item Characterized, modified and improved existing Audio DSP Algorithms
\item Implemented DSP algorithms within existing C codebase
\item Maintain and expand on Audio CODEC binaries and their respective Linux Kernel modules
\end{itemize}
}

\cventry{Jan - Dec 2012/2013}
{Research Assistant}
{Thesis}
{Texas State University}
{}
{
Spearheaded research on a relatively young transport
  protocol (SCTP).
\begin{itemize}
\item Performed a extensive literature review on current SCTP
  research.
\item Designed and developed testing environment running on Linux that
  automated Kernel Module compilation, loading/unloading and logged
  relevant SCTP transfer parameters.
\item Performed in-depth analysis of SCTP retransmission parameters
  including: Return Time Out, Round Trip Time and Return Time Out
  Minimum.
\item Performed in-depth analysis of the effect of network parameters
  on SCTP performance including: Packet Loss, Packet Spacing and
  Packet Delay.
\item Evaluated a proposed adaptive algorithm for managing the Return
  Time Out Minimum SCTP parameter.
\item Discovered bug in SCTP Linux Kernel Module and submitted fix to
  developers mailing list.
\end{itemize}
}

\cventry{Jul - Sep 2012}
{GNURadio Module for Custom SDR Platform}
{FlexRadio Systems}
{Austin, TX}
{}
{
Developed a GNURadio
  module to control, receive and display samples from a custom high
  bandwidth radio receiver.
\begin{itemize}
\item Created C Library that used TCP sockets to send and receive
  custom control structures from SDR hardware and used UDP sockets to
  receive raw IQ samples from SDR hardware in VITA49 format over
  10GbE.
\item Created Python wrappings for C library using SWIG.
\item Created graphical block for GNURadio Companion using XML.
\end{itemize}
}

\cventry{Jun - Aug  2011}
{Speech Detector}
{Flex-Radio Systems}
{Austin, TX}
{}
{
Created non-computationally intensive speech detector
  for existing Software Designed Radio System based on Motorola's
  MICOM circuit \cite{VOXPaper}.
\begin{itemize}
\item Received and analyzed schematic for an analog speech detection
  system developed in the 70s for HAM Radio operators.
\item Modeled analog circuit using Multisim with a speech audio file
  as input.
\item Created a digital model of the analog speech detector using
  Simulink.
\item Conducted a literature review to find applicable approaches for
  speech detection in audio.
\item Combined features from the analog circuit and digital approaches
to create a robust speech detection system that maintained very low
computation overhead.
\end{itemize}
}

%\subsection{School Related}
\cventry{Jan - Apr 2010/2011}{Autonomous Robot}{IEEE Student
  Chapter}{}{}{Designed and constructed small autonomous robots (from scratch) for
  the IEEE Region 5 Student Robotics Competion. 
\begin{itemize}
\item Programmed entirely in assembly (68k) on Freescale 68HC12.
\item Small mobile robots completed specific taks, including locating,
  analyzing and transporting objects while avoiding obstacles.
\item Used a variety of sensors (IR, UltraSonic, Pressure) to complete
  tasks.
\end{itemize}}

\cventry{Feb - Mar 2011}{CPU Pipeline Simulator}{Class Project}{}{}{
  Implemented a MIPS interpreter with a simulation of CPU Pipeline
  execution including collisions in C. }

\cventry{Jan - Dec 2010}
{Real-Time Powerline DAQ}
{Capstone Project}
{}
{}
{
Modified existing LabView DAQ System and created new
  system in order to capture, save and display real-time waveforms
  using Agilent U2353A DAQ.
\begin{itemize}
\item Designed as Queued-State Machine.
\item Used primarily to capture voltage and current information from
live power-line feeds.
\item Allos the implementation of post caputre processing on live
  streams using Matlab.
\end{itemize}
}

\cventry{Aug - Dec 2010}
{Teaching Assistant}
{``Circuits and Devices''}
{}
{}
{Helped manage class lab section.}

\section{Activities}
\cventry{2010 - 2011}{President}{IEEE Student Branch}{}{}{
  \begin{itemize}
  \item Awarded Outstanding Small Student Branch in Region 5 during term.
  \item Represented the TxState IEEE Student Branch in all external affairs
  \item Hosted ``Future City'' Competition for middle school kids
  \item Founding member of Student Branch
  \end{itemize} }

\cventry{2010 - 2011}{Vice-President}{Water Polo Club}{}{}{
  \begin{itemize}
  \item Help manage Club's operations and representation within school system.
  \end{itemize} }

\renewcommand{\listitemsymbol}{-~}            % change the symbol for lists

\end{document}
